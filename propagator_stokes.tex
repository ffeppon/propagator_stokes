
\documentclass[a4paper,10pt,reqno]{amsart}

\usepackage[english]{babel}
\usepackage[margin=2cm]{geometry}
\usepackage[utf8]{inputenc}

\usepackage{subcaption}
\usepackage[shortlabels]{enumitem}
\setlist[enumerate]{label=(\roman*)}

% Packages and macros go here
\usepackage{comment}
\usepackage{tikz}
\usetikzlibrary{arrows.meta}
\usetikzlibrary{shapes.arrows}
\usepackage{pgfplots}
\usepackage{epstopdf}
\pgfplotsset{compat=1.17}

\usepackage{stmaryrd}
\usepackage{amsfonts}
\usepackage{graphicx}
\usepackage{epstopdf}
\usepackage{color}

\SetSymbolFont{stmry}{bold}{U}{stmry}{m}{n}
\usepackage{amsmath}
\usepackage{amssymb}
\usepackage{float}%For force table and figure position
\usepackage{amsthm}
\usepackage{bm}
\usepackage[colorlinks=true]{hyperref}
\definecolor{GreenLink}{RGB}{0,128,0}
\hypersetup{
      final,
        linktoc=all,
    colorlinks,
        allcolors=GreenLink
}
\numberwithin{equation}{section}
\usepackage{mathrsfs}
\usetikzlibrary{matrix,positioning,decorations.pathreplacing,calc}
\makeatletter
\def\grd@save@target#1{%
  \def\grd@target{#1}}
\def\grd@save@start#1{%
  \def\grd@start{#1}}
\tikzset{
  grid with coordinates/.style={
    to path={%
      \pgfextra{%
        \edef\grd@@target{(\tikztotarget)}%
        \tikz@scan@one@point\grd@save@target\grd@@target\relax
        \edef\grd@@start{(\tikztostart)}%
        \tikz@scan@one@point\grd@save@start\grd@@start\relax
        \draw[minor help lines] (\tikztostart) grid (\tikztotarget);
        \draw[major help lines] (\tikztostart) grid (\tikztotarget);
        \grd@start
        \pgfmathsetmacro{\grd@xa}{\the\pgf@x/1cm}
        \pgfmathsetmacro{\grd@ya}{\the\pgf@y/1cm}
        \grd@target
        \pgfmathsetmacro{\grd@xb}{\the\pgf@x/1cm}
        \pgfmathsetmacro{\grd@yb}{\the\pgf@y/1cm}
        \pgfmathsetmacro{\grd@xc}{\grd@xa + \pgfkeysvalueof{/tikz/grid with coordinates/major step}}
        \pgfmathsetmacro{\grd@yc}{\grd@ya + \pgfkeysvalueof{/tikz/grid with coordinates/major step}}
        \foreach \x in {\grd@xa,\grd@xc,...,\grd@xb}
        \node[anchor=north] at (\x,\grd@ya) {\pgfmathprintnumber{\x}};
        \foreach \y in {\grd@ya,\grd@yc,...,\grd@yb}
        \node[anchor=east] at (\grd@xa,\y) {\pgfmathprintnumber{\y}};
      }
    }
  },
  minor help lines/.style={
    help lines,
    step=\pgfkeysvalueof{/tikz/grid with coordinates/minor step}
  },
  major help lines/.style={
    help lines,
    line width=\pgfkeysvalueof{/tikz/grid with coordinates/major line width},
    step=\pgfkeysvalueof{/tikz/grid with coordinates/major step}
  },
  grid with coordinates/.cd,
  minor step/.initial=.2,
  major step/.initial=1,
  major line width/.initial=2pt,
}

\tikzset{
    style1/.style={
      matrix of math nodes,
      every node/.append style={text width=#1,align=center,minimum height=5ex},
      nodes in empty cells,
      left delimiter=[,
      right delimiter=],
      },
    style2/.style={
      matrix of math nodes,
      every node/.append style={text width=#1,align=center,minimum height=5ex},
      nodes in empty cells,
      left delimiter=\lbrace,
      right delimiter=\rbrace,
      }
}
\makeatother% Load TikZ
\renewcommand{\bar}{\overline}
\newcommand{\N}{\mathbb{N}}
\newcommand{\Z}{\mathbb{Z}}
\newcommand{\Q}{\mathbb{Q}}
\renewcommand{\tilde}{\widetilde}
\newcommand{\R}{\mathbb{R}}
\newcommand{\C}{\mathbb{C}}
\newcommand{\F}{\mathbb{F}}
\newcommand{\FF}[1]{{\color{blue}FF: #1}}
\newcommand{\D}{\ensuremath \,\mathrm{d}}
\newcommand{\y}{{\bm y}} \newcommand{\z}{\bm z} 
\newcommand{\<}{\ensuremath \leq}
\renewcommand{\>}{\ensuremath \geq}
\newcommand{\Sym}{\mathrm{Sym}}
\renewcommand{\skew}{\mathrm{skew}}
\newcommand{\grad}{\mathrm{\textbf{grad}}}
\newcommand{\Ker}{\mathrm{Ker}}
\newcommand{\Ran}{\mathrm{Ran}}
\let\oldv\v
\let\oldn n
\renewcommand{\t}{\tilde}
\renewcommand{\v}{{\bm v}}
\newcommand{\w}{{\bm w}}
\newcommand{\x}{{\bm {x}}}
\newcommand{\half}{\frac{1}{2}}
\newcommand{\Tr}{\mathrm{Tr}}
\newcommand{\DD}{\mathrm{D}}
%\usepackage{subcaption}
\renewcommand{\arraystretch}{1.5}
%%LOCAL SHORTCUTS
\let\oldu\u
\renewcommand{\u}{{\bm u}}
\newcommand{\f}{\bm F}
\renewcommand{\div}{\mathrm{div}}
\newcommand{\In}{\textrm{ in }}
\newcommand{\On}{\textrm{ on }}
\newcommand{\n}{{\bm  n}}
\newcommand{\g}{{\bm g}}
\newcommand{\h}{{\bm h}}
\let\oldtau\tau
\renewcommand{\tau}{\oldtau}
%\let\oldtheta\theta
%\renewcommand{\theta}{{\bm \oldtheta}}
\renewcommand{\r}{\bm r}
\newcommand{\partialn}[1]{\frac{\partial #1}{\partial n}}
\newcommand{\ieps}{\epsilon^{-1}}
\newcommand{\e}{{\bm e}}
\newcommand{\At}{\textrm{ at }}
\renewcommand{\L}{\mathcal{L}}
\newcommand{\norm}[1]{\left|\left|#1\right|\right|}

\newcommand{\vect}{\mathrm{span}}
%\let\oldxi\xi
%\renewcommand{\xi}{{\bm \oldxi}}
\newcommand{\CC}{{\bm C}}
%\let\oldmu\mu
%\renewcommand{\mu}{{\bm \oldmu}}
\newcommand{\pardiff}[2]{\frac{\partial #1}{\partial #2}}
\usepackage[nameinlink,capitalize]{cleveref}
\crefname{figure}{Figure}{Figs.}
\crefname{equation}{}{}
\usepackage{amsopn}
\DeclareMathOperator{\diag}{diag}
\DeclareMathOperator{\curl}{curl}
\DeclareMathOperator{\bcurl}{\mathbf{curl}}
\DeclareMathOperator{\diam}{diam}
\DeclareMathOperator{\dist}{dist}
\newcommand{\If}{\textrm{ if }}
\newcommand{\I}{I}
\theoremstyle{definition}
\newtheorem{proposition}{Proposition}[section]
\newtheorem{corollary}{Corollary}[section]
\newtheorem{lemma}{Lemma}[section]
\theoremstyle{definition}
\newtheorem{definition}{Definition}[section]
\newtheorem{conjecture}{Conjecture}[section]
\theoremstyle{remark}
\newtheorem{remark}{Remark}[section]
\newcommand{\SSigma}{\overline{\Sigma}}
\newcommand{\Chi}{\mathcal{X}}
\renewcommand{\setminus}{-}
\renewcommand{\paragraph}[1]{\noindent\underline{#1}}
\newcommand{\etaPY}{\eta^{-1}P \setminus T}
\newcommand{\B}{\mathbb{B}}
\RequirePackage{silence}
\WarningFilter{remreset}{The remreset package}
\usepackage{thmtools}
\declaretheoremstyle[
spaceabove=6pt, spacebelow=6pt,
headfont=\normalfont\bfseries,
notefont=\mdseries, notebraces={(}{)},
bodyfont=\normalfont,
postheadspace=0.6em,
headpunct={}
]{mystyle}
\declaretheorem[style=mystyle, name={},preheadhook={\renewcommand{\thehyp}{(H{\arabic{hyp}})}}]{hyp}
\crefname{hyp}{}{}
\Crefname{hyp}{}{}
\newcommand{\A}{\mathcal{A}}
\newcommand{\CChi}{{\bm\Chi}}

%\newcommand{\PPsi}{{\bm\Psi}}
%\newcommand{\PPhi}{{\bm\Phi}}

 %New line after subsection or subsubsection
\makeatletter
\def\subsection{\@startsection{subsection}{3}%
  \z@{.5\linespacing\@plus.7\linespacing}{.5\linespacing}%
  {\bf}}
\makeatother

\makeatletter
\def\subsubsection{\@startsection{subsubsection}{3}%
  \z@{.5\linespacing\@plus.7\linespacing}{.5\linespacing}%
  {\it}}
\makeatother


\makeatletter
\renewcommand*{\fps@figure}{htp}
\makeatother
\let\Cap\relax
\DeclareMathOperator{\Cap}{cap}
%\newcommand{\Cap}{\mathrm{cap}}
\renewcommand{\a}{{\bm a}}
\newcommand{\s}{{\bm s}}
\renewcommand{\setminus}{\backslash}
\renewcommand{\b}{{\bm b}}
\newcommand{\M}{{\bm M}}
\DeclareMathOperator{\Var}{Var}
\newcommand{\DN}{D_{N,s}}
\renewcommand{\F}{{\bm F}}
\newcommand{\IN}{{\rm in}}
\newcommand{\out}{{\rm out}}
\newcommand{\wall}{{\rm wall}}
\newcommand{\tot}{{\rm tot}}
\newcommand{\Peta}{{P\setminus(\eta T)}}
\newcommand{\per}{{\rm per}}
\newcommand{\loc}{{\rm loc}}
\newcommand{\ii}{{\rm i}}
\renewcommand{\t}{t}
\newcommand{\eff}{{\rm eff}}
\usetikzlibrary{matrix,positioning,decorations.pathreplacing,calc}
\usetikzlibrary{decorations.pathreplacing,decorations.markings}
\usetikzlibrary{arrows.meta}
\tikzset{
  % style to apply some styles to each segment of a path
  on each segment/.style={
    decorate,
    decoration={
      show path construction,
      moveto code={},
      lineto code={
        \path [#1]
        (\tikzinputsegmentfirst) -- (\tikzinputsegmentlast);
      },
      curveto code={
        \path [#1] (\tikzinputsegmentfirst)
        .. controls
        (\tikzinputsegmentsupporta) and (\tikzinputsegmentsupportb)
        ..
        (\tikzinputsegmentlast);
      },
      closepath code={
        \path [#1]
        (\tikzinputsegmentfirst) -- (\tikzinputsegmentlast);
      },
    },
  },
  % style to add an arrow in the middle of a path
  mid arrow/.style={postaction={decorate,decoration={
        markings,
        mark=at position .5 with {\arrow[#1]{stealth}}
      }}},
}
\makeatletter
\def\grd@save@target#1{%
  \def\grd@target{#1}}
\def\grd@save@start#1{%
  \def\grd@start{#1}}
\tikzset{
  grid with coordinates/.style={
    to path={%
      \pgfextra{%
        \edef\grd@@target{(\tikztotarget)}%
        \tikz@scan@one@point\grd@save@target\grd@@target\relax
        \edef\grd@@start{(\tikztostart)}%
        \tikz@scan@one@point\grd@save@start\grd@@start\relax
        \draw[minor help lines] (\tikztostart) grid (\tikztotarget);
        \draw[major help lines] (\tikztostart) grid (\tikztotarget);
        \grd@start
        \pgfmathsetmacro{\grd@xa}{\the\pgf@x/1cm}
        \pgfmathsetmacro{\grd@ya}{\the\pgf@y/1cm}
        \grd@target
        \pgfmathsetmacro{\grd@xb}{\the\pgf@x/1cm}
        \pgfmathsetmacro{\grd@yb}{\the\pgf@y/1cm}
        \pgfmathsetmacro{\grd@xc}{\grd@xa + \pgfkeysvalueof{/tikz/grid with coordinates/major step}}
        \pgfmathsetmacro{\grd@yc}{\grd@ya + \pgfkeysvalueof{/tikz/grid with coordinates/major step}}
        \foreach \x in {\grd@xa,\grd@xc,...,\grd@xb}
        \node[anchor=north] at (\x,\grd@ya) {\pgfmathprintnumber{\x}};
        \foreach \y in {\grd@ya,\grd@yc,...,\grd@yb}
        \node[anchor=east] at (\grd@xa,\y) {\pgfmathprintnumber{\y}};
      }
    }
  },
  minor help lines/.style={
    help lines,
    step=\pgfkeysvalueof{/tikz/grid with coordinates/minor step}
  },
  major help lines/.style={
    help lines,
    line width=\pgfkeysvalueof{/tikz/grid with coordinates/major line width},
    step=\pgfkeysvalueof{/tikz/grid with coordinates/major step}
  },
  grid with coordinates/.cd,
  minor step/.initial=.2,
  major step/.initial=1,
  major line width/.initial=2pt,
}

\tikzset{
    style1/.style={
      matrix of math nodes,
      every node/.append style={text width=#1,align=center,minimum height=5ex},
      nodes in empty cells,
      left delimiter=[,
      right delimiter=],
      },
    style2/.style={
      matrix of math nodes,
      every node/.append style={text width=#1,align=center,minimum height=5ex},
      nodes in empty cells,
      left delimiter=\lbrace,
      right delimiter=\rbrace,
      }
}
\makeatother% Load TikZ
\DeclareMathOperator{\sign}{sign}
\newcommand{\bl}{{\rm bl}}
\renewcommand{\epsilon}{\varepsilon}

%%%% To get table of contents %%%%%%%%
\setcounter{tocdepth}{2}% to get subsubsections in toc
\usepackage{bibentry}

\let\oldtocsection=\tocsection

\let\oldtocsubsection=\tocsubsection
\pagestyle{plain}
\let\oldtocsubsubsection=\tocsubsubsection

\renewcommand{\tocsection}[2]{\hspace{0em}\oldtocsection{#1}{#2}\textbf}
\renewcommand{\tocsubsection}[2]{\hspace{1em}\oldtocsubsection{#1}{#2}}
\renewcommand{\tocsubsubsection}[2]{\hspace{2em}\oldtocsubsubsection{#1}{#2}}
\renewcommand{\paragraph}[1]{\noindent\underline{#1}}

\begin{document}
\title{ Stokes flows in semi-infinite periodic strips }
\author{F. Feppon$^{1}$, S. Fliss$^{2}$}
\thanks{\hspace*{0.3em}\textsuperscript{$*$} Corresponding author. Email:
\texttt{florian.feppon@kuleuven.be}. \
\hspace*{\parindent}
}

\maketitle


%%%%%%%%%%%%%%%%%%%%%%%%%%%%%%%%%%%%%%%%%%%%%%%%%%%%%%%%%%%%%%%%%%%%%%%%%%%%%%%%%%%%%%%%%%%%%%%%
\vspace{-2em}
\begin{center}
\emph{\textsuperscript{1} KU Leuven, Department of Computer Science, NUMA Unit,
    Belgium.}\\
    \emph{\textsuperscript{2} ENSTA Paris, UMR 7231 CNRS-INRIA-ENSTA,
    France.}
\end{center}
%%%%%%%%%%%%%%%%%%%%%%%%%%%%%%%%%%%%%%%%%%%%%%%%%%%%%%%%%%%%%%%%%%%%%%%%%%%%%%%%%%%%%%%%%%%%%%%%

\begin{abstract} 
\end{abstract} 
\medskip  
\noindent \textbf{Keywords.} keyword1, keyword2. 

\medskip
\noindent \textbf{AMS Subject classifications.}  \par
%35B34 resonances in the context of PDEs
%35B40   	Asymptotic behavior of solutions to PDEs
%45M05   	Asymptotics of solutions to integral equations
%35J05   	Laplace operator, Helmholtz equation (reduced wave equation), Poisson equation %%%%%%%%%%%%%%%%%%%%%%%%%%%%%%%%%%%%%%%%%%%%%%%%%%%%%%%%%%%%%%%%%%%%%%%%%%%%%%%%%%%%%%%%%%%%%%%%
% 35B10   	Periodic solutions to PDEs
\medskip
\bigskip
\hrule
\tableofcontents
\vspace{-0.5cm}
\hrule
\medskip
\bigskip
%%%%%%%%%%%%%%%%%%%%%%%%%%%%%%%%%%%%%%%%%%%%%%%%%%%%%%%
\section{Introduction}
    
We consider the infinite strip $\mathcal{B}$ and the semi-infinite strip $\mathcal{B}_+$ illustrated    
on \cref{fig:yad9r,fig:2w1yx} below.
The infinite strip $\mathcal{B}$ is defined by
\[
\mathcal{B}:= \bigcup_{i\in\Z} Y_i \text{ with }Y_i:=i\e_1 + Y,
\]     
 obtained by juxtaposing periodically
the unit cell $Y$.     
The unit cell $Y$ is represented on \cref{fig:1e8x0}:    
it is defined as $Y=P\setminus \bar T$, where $P=(0,1)^{d}$     
and $T\subset P$ is an open subdomain representing     
the solid part.  We denote by   
$\Gamma_{x_1}$ the section  
\[
    \Gamma_{x_1}:=\{ (x_1,x')\,|\, (x_1,x')\in \mathcal{B} \}.
\] 

\begin{figure}
    \centering  
    \begin{tikzpicture}[scale=2.1]
        \foreach \n in {0,1,...,7} {
        \draw[fill=black!10!white] plot[smooth cycle, tension=2] coordinates {
            ($(0.4+\n,0.3)$) ($(0.7+\n,0.5)$) ($(0.5+\n,0.7)$)  };
    }   
    \foreach \n in {1,2,...,8}{ 
        \pgfmathsetmacro{\m}{int(\n-4)};
        \draw ($(\n-0.5,0.5)$) node {$T_{\m}$};	
        \draw ($(\n-0.85,0.85)$) node {$Y_{\m}$};	
    }
    \foreach \n in {1,2,...,7} {    
        \pgfmathsetmacro{\m}{int(\n-3)};
        \draw ($(\n,0.04)$) node[below] {$\Gamma_{\m}$};	
        \draw[dotted] ($(\n,0)$)--($(\n,1)$); 
    }
    \draw (3.5,0) node[below] {$\Gamma_{\wall}$}; 
    \draw (3.5,1) node[above] {$\Gamma_{\wall}$}; 
    \draw (0,0) -- (8,0);   
    \draw (8,1) -- (0,1);
    \end{tikzpicture}
    \caption{The infinite strip $\mathcal{B}$.}
    \label{fig:yad9r}
\end{figure}    

\begin{figure}
    \centering  
    \begin{tikzpicture}[scale=1.8]
    \draw[dashed] (2.4,0) -- (2.4,1); 
    \draw (2.4,0) node[below] {$\Gamma_{x_1}$}; 
        \foreach \n in {0,1,...,7} {
        \draw[fill=black!10!white] plot[smooth cycle, tension=2] coordinates {
            ($(0.4+\n,0.3)$) ($(0.7+\n,0.5)$) ($(0.5+\n,0.7)$)  };
    }   
    \foreach \n in {1,2,...,8}{ 
        \pgfmathsetmacro{\m}{int(\n-1)};
        \draw ($(\n-0.5,0.5)$) node {$T_{\m}$};	
    }
    \foreach \n in {1,2,...,7} {    
        \draw[dotted] ($(\n,0)$)--($(\n,1)$); 
    }
    \draw (4,0) node[below] {$\Gamma_{\wall,+}$};
    \draw (4,1) node[above] {$\Gamma_{\wall,+}$};
    \draw (0,0) node[below] {$\Gamma_{0}$};
    \draw (1,0) node[below] {$\Gamma_{1}$};
    \draw (0,0) -- (8,0);   
    \draw (8,1) -- (0,1);
    \draw (0,0) -- (0,1);
    %\foreach \y in {0,0.1,...,1} {  
    %    \pgfmathsetmacro{\yy}{4*(\y*(1-\y)*(\y-0.5))}; 
    %    \draw[->,>=stealth] (\yy,\y) -- (0,\y);
    %}
    \end{tikzpicture}
    \caption{The semi-infinite strip $\mathcal{B}_{+}$.}
    \label{fig:2w1yx}
\end{figure}    

\begin{figure}
    \centering  
    \begin{tikzpicture}[scale=4]
        \draw (0,0) rectangle (1,1); 
        \draw[dashed] (0.25,0) -- (0.25,1);   
        \draw (0.25,0.8) node[right] {$\Gamma_{x_1}$}; 
        \draw (0.25,0) node[below] {$(x_1,0)$}; 
        \draw[fill=black!10!white] plot[smooth cycle, tension=2] coordinates { (0.2,0.3) (0.7,0.5) (0.5,0.7)  };
        \node[above left] at (1,0.1) {$Y=P\setminus T$};
        \node[above] at (0.5,1) {$\Gamma_w$};
        \node[below] at (0.5,0) {$\Gamma_w$};
        \node[above left] at (1,0) {$P=(0,1)^{d}$};
        \node at (0.5,0.5) {$T$};
        \node[left] at (0,0.5) {$\Gamma_{0}$}; 
        \node[right] at (1,0.5) {$\Gamma_{1}$}; 
    \end{tikzpicture}
    \caption{The unit cell $Y=P\setminus \bar T$, the wall boundaries $\Gamma_w$ and the
    section $\Gamma_{x_1}$.}
    \label{fig:1e8x0}
\end{figure}
Let us denote by $\dotH$ the space   of inlet
velocities with zero flux.
\[
\dotH:=\left\{ \v\in H^{\frac{1}{2}}_{00}(\Gamma_0)\, | \,
\int_{\Gamma_0}\v\cdot\e_1\D \sigma =0\right\}
\] 
For a given boundary datum $\v_0\in \dotH $, there exists a unique solution $(\v,q)\in
H^{1}(\mathcal{B}_+)\times L^{2}(\mathcal{B}_+)$ to the
following Stokes problem:  
\begin{equation}
\label{eqn:235q7}
\left\{\begin{aligned}
        -\Delta \v+\nabla q& = 0\In \mathcal{B}_+,\\    
        \div(\v) &= 0 \In \mathcal{B}_+,\\  
        \v& = 0 \On \partial \mathcal{B}_+\backslash \Gamma_0,\\    
        \v &= \v_0 \On \Gamma_0.
\end{aligned}
\right.
\end{equation}
This is established in \cite{nazarov_navier-stokes_2000-1,feppon_asymptotic_2024}.  
    
The questions we want to investigate are:   
\begin{itemize}
    \item Can we propose a well-posedness theory for the Stokes operator in the
        infinite strip $\mathcal{B}$ capturing various types of exponential
        growth/decay in the strip? This seems to have been done extensively in
        \cite{nazarov_navier-stokes_2000-1}.    
    \item For the solution ($(\v,q)$) to the Stokes problem in the semi-infinite strip
        $\mathcal{B}_+$, can we prove, for any $M\in\N$, an asymptotic expansion of
        the form    
        \begin{equation}
        \label{eqn:chwra}
        \v(x_1,x')=\sum_{i=0}^{M}\sum_{k=0}^{k_i} x_1^{k}e^{-\gamma_i
        x_1}\v_{i,k}(x_1,x')+\v_M \text{ with }\v_M \in
        W_\beta(\mathcal{B}_+),\forall \beta\in \C \text{ with
        }\Re(\beta)>\Re(\gamma_M),
        \end{equation}
        \begin{equation}
        \label{eqn:0ec8r}
        q(x_1,x')=\sum_{i=0}^{M}\sum_{k=0}^{k_i} x_1^{k}e^{-\gamma_i
        x_1}q_{i,k}(x_1,x')+q_M \text{ with }\v_M \in
        W_\beta(\mathcal{B}_+),\forall \beta\in \C \text{ with
        }\Re(\beta)>\Re(\gamma_M),
        \end{equation}
        where $(\v_{i,k},q_{i,k}$ are smooth periodic vector and scalar fields,     
        $W_{\beta}(\mathcal{B}_+)=\{ \w\,|\,e^{\beta x_1}\w \in
        H^{1}(\mathcal{B}_+)\}$ is a space of functions decaying exponentially fast
        at infinity.
        Such type of result can hopefully be obtained with Kondratyev's theory
        \cite{kozlov_elliptic_1997,nazarov_elliptic_2011}. Expansions of the form of
        \cref{eqn:chwra,eqn:0ec8r} have been obtained for solutions to wave problems
        in periodic structures in
        \cite{hohage_riesz_2013,bourgeois_well-posedness_2018}.
    \item Can we construct a numerical procedure based on propagation operators     
        as in
        \cite{fliss_dirichlet--neumann_2013,fliss_dirichlet--neumann_2021}? Can we
        devise efficient schemes to compute limiting constants for the pressure?    
        Formal numerical schemes have been proposed in the heat transfer engineering
        community \cite{buckinx_macro-scale_2022,vangeffelen_developed_2023}.
\end{itemize}
\section{Numerical solution of the semi-infinite strip problem based on the
propagator method}
    
In all what follows, we denote by $\mathcal{V}\,:\,\dotH\to H^{1}(\mathcal{B}_+)$ and
$\mathcal{Q}\,:\ \dotH\to L^{2}(\mathcal{B}_+)$ the solution
operators for \cref{eqn:235q7}: 
\[
    \mathcal{V}[\v_0]:=\v \text{ and }\mathcal{Q}[\v_0]:=p.
\] 
We denote by $\tau_{x_1}$ the translation     
\[
\tau_{x_1}(x_1,x'):=(x_1+1,x').
\] 
We shall denote by $\tau$   
 the translation operator 
\[
    \tau[\pphi](x_1,x'):=\pphi\circ \tau_{x_1}^{-1}(x_1,x')=\pphi(x_1-1,x'),    
\]                        %
which translates a function $\pphi$ defined on $\Gamma_0$ on a function defined on
$\Gamma_1$, 
where the precise argument and image spaces will be clear from the context. 
\begin{definition}        %
    We call `propagator operator' the mapping   
    \begin{equation}
    \label{eqn:6gnkg}   
    \begin{aligned}
        \mathcal{P}\,:\,& \dotH&  \to & \dotH \\  
        & \v_0 &  \mapsto & (\mathcal{V}[\v_0])|_{\Gamma_1}\circ \tau_{x_1},
    \end{aligned}
    \end{equation}
    where $\v$ is the unique solution to the problem \cref{eqn:235q7}.
\end{definition}
It follows from interior regularity estimates that $\mathcal{P}$ is compact. We can
also extend $\mathcal{P}$ on the whole $H_{00}^{\frac{1}{2}}(\Gamma_0)$.
Let $(\Chi,\alpha)$ the solution to the periodic cell problem   
\[
\left\{\begin{aligned}
        -\Delta \Chi+\nabla\alpha &=  \e_1 \In Y,\\ 
        \Chi &= 0\On \partial T,\\  
        \Chi,\alpha &\text{ are $P$--periodic}.
\end{aligned}
\right.
\] 
We have the space decomposition     
\[
    H_{00}^{\frac{1}{2}}(\Gamma_0) = \vect(\Chi)\oplus \dotH.
\] 
Indeed,     
\[
\forall \w\in H_{00}^{\frac{1}{2}}(\Gamma_0),\quad
\w=\underbrace{\frac{\int_{\Gamma_0}\w\cdot\n\D\sigma}{\int_{\Gamma_0}\Chi\cdot\n\D
\sigma} \Chi}_{\in \vect(\Chi)}
+\underbrace{\left( \w-\frac{\int_{\Gamma_0}\w\cdot\n\D\sigma}{\int_{\Gamma_0}\Chi\cdot\n\D
\sigma} \Chi \right)}_{\in \dotH}.
\] 
Thus, we can extend $\mathcal{P}$ to $H^{\frac{1}{2}}_{00}(\Gamma_0)$ by setting     
\[
    \mathcal{P}[\Chi|_{\Gamma_0}] = \Chi|_{\Gamma_0}.
\] 
And similarly, we can extend $\mathcal{V}$ to $H^{\frac{1}{2}}_{00}(\Gamma_0)$ by setting     
\[
    \mathcal{V}[\Chi_{|\Gamma_0}](x_1,x')=\Chi(x_1,x'),\quad \forall (x_1,x')\in
    \mathcal{B}_+.
\] 
By uniqueness and periodicity of the strip, it holds then, for any $\v_0\in
H^{\frac{1}{2}}_{00}(\Gamma_0)$: 
\begin{equation}
\label{eqn:o0rih}
    \mathcal{V}[\v_0](x_1+1,x')=\mathcal{V}\left[\mathcal{P}[\v_0]\right](x_1,x'),\quad
    \forall (x_1,x')\in \mathcal{B}_+.
\end{equation}
In order to obtain a functional equation for $\mathcal{P}$, we need the following
lemma.
\begin{lemma}
    \label{lem:3a1i0}
    Let $(\w_0,r_0)\in H^{1}(Y_0,\R^{d})\times L^{2}(Y_0)$ and $(\w_1,r_1)\in
    H^{1}(Y_1,\R^{d})\times L^{2}(Y_1)$ satisfying  
    \begin{equation}
    \label{eqn:he7zb}   
    \left\{\begin{aligned}
            -\Delta \w_0+\nabla r_0 &=0 \In Y_0,\\  
            \div(\w_0) &= 0\In Y_1,
    \end{aligned}\right. \qquad
    \left\{\begin{aligned}
            -\Delta \w_1+\nabla r_1 &=0 \In Y_1,\\  
            \div(\w_1) &= 0\In Y_1.
    \end{aligned}\right.
    \end{equation}
    Then the velocity and pressure field $(\w,r)$ defined in $Y_0\cup Y_1$ by   
    \begin{equation}
    \label{eqn:topkr}
        (\w,r)|_{Y_0}:=(\w_0,r_0) \text{ and }(\w_,r)|_{Y_1}:=(\w_1,r_1)
    \end{equation}
    satisfies   
    \begin{equation}
    \label{eqn:s9bzn}   
    \left\{\begin{aligned}
            -\Delta \w+\nabla r&=0\In Y_0\cup Y_1,\\    
            \div(\w)& = 0 \In Y_1
    \end{aligned}
    \right.
    \end{equation}
    if and only if we have continuity of the function and of the conormal derivative
    on the interface $\Gamma_1$:
    \begin{equation}
    \label{eqn:8zwjf}
    \w_0|_{\Gamma_1}=\w_1|_{\Gamma_1} \text{ and
    }\nabla\w_0\e_1-r_0\e_1\vert_{\Gamma_1}=\nabla \w_1\e_1-r_1\e_1\vert_{\Gamma_1}.
    \end{equation}
\end{lemma}
\begin{proof}
    Consider the problem    
    \renewcommand{\w}{\widehat{\bm w}}
    \[
    \left\{\begin{aligned}
            -\Delta \w +\nabla \hat r &= 0 \In Y_0\cup Y_1,\\    
            \div(\w)& = 0 \In Y_0\cup Y_1,\\    
            \w &=  \w_0 \On \partial Y_0\backslash \Gamma_1,\\  
            \w &=  \w_1 \On \partial Y_1 \backslash\Gamma_1.
    \end{aligned}
    \right.
    \] 
    The variational formulation associated to this problem is   
    \begin{multline}
    \text{find }(\w,\hat r)\in H^{1}(Y_0\cup Y_1,\R^{d})\times L^{2}(Y_0\cup Y_1) \text{
    such that }\w=\w_0 \On \partial Y_0\backslash\Gamma_1,\; \w=\w_1 \On \partial
    Y_1\backslash\Gamma_1 \text{ and }\\    
    \int_{Y_0\cup Y_1}\left(\nabla\w:\nabla \w'-\hat r \div(\w') -
    r'\div(\w)\right) \D x =0, \\  \forall (\w',r')\in H^{1}(Y_0\cup Y_1)\times
    L^{2}(Y_0\cup Y_1) \text{ with }\w'=0\On \partial (Y_0\cup Y_1).
    \end{multline} 
    It is readily seen that the velocity and pressure field $(\w,\hat r)$ defined by
    \cref{eqn:topkr} satisfies  
    \[
    \int_{Y_0\cup Y_1}(\nabla\w:\nabla \w'-\hat r\div(\w')-r'\div'(\w))\D x =
    \int_{\Gamma_1}\left[ (\nabla\w_0\e_1-r_0\e_1)-(\nabla \w_1\e_1-r_1\e_1)
    \right]\cdot \w'\D \sigma.
    \] 
    \renewcommand{\w}{{\bm w}}     
    Therefore, $(\w,r)$ satisfies \cref{eqn:s9bzn} if and only if \cref{eqn:8zwjf} is
    satisfied. Then $\w=\hat \w$ and there exists a constant $C$ such that $r=\hat
    r+C$. 
\end{proof}
\subsection{Characterization of the propagation operator: a first Riccati equation}
In order to characterize the propagator operator through a Riccati equation, we
follow the technique introduced in \cite{fliss_dirichlet--neumann_2013}. Let us
introduce the operators $\mathcal{E}_0\,:\, \dotH\to H^{1}(Y_0,\R^{d})$, 
$\mathcal{E}_1\,:\,\dot H^{\frac{1}{2}}_{00}(\Gamma_1)\to H^{1}(Y_0,\R^{d})$,   
$\mathcal{A}_0\,:\, \dotH\to L^{2}_0(Y_0)$ and   
$\mathcal{A}_1\,:\, \dot H^{\frac{1}{2}}_{00}(\Gamma_1)\to L^{2}_0(Y_0)$ the
operators defined by    
\begin{equation}
    \label{eqn:2a0sn}\mathcal{E}_0[\pphi_0]:=\v_0 \text{ and
    }\mathcal{E}_1[\pphi_1]:=\v_1 
\end{equation}
\begin{equation}
\label{eqn:qywvj}   
\mathcal{A}_0[\pphi_0]:=\alpha_0 \text{ and }\mathcal{A}_1[\pphi_1]:=\alpha_1.
\end{equation}
where $(\v_0,\alpha_0)$ and $(\v_1,\alpha_1)$ are the unique solutions to the Stokes
problems
\[
    \left\{\begin{aligned}
    -\Delta  \v_0 + \nabla \alpha_0 &= 0 \In Y_0,\\  
    \div(\v_0) &= 0 \In Y_0,\\  
    \v_0 &= \bm\phi_0 \On \Gamma_0,\\    
    \v_0 &=  0 \On \partial Y_0\backslash\Gamma_0,  \\
    \int_{Y_0}\alpha_0\D x &= 0,
    \end{aligned}
    \right.\qquad 
\left\{\begin{aligned}
    -\Delta  \v_1 + \nabla \alpha_1 &= 0 \In Y_0,\\  
    \div(\v_1) &= 0 \In Y_0,\\  
    \v_1 &= \bm\phi_1 \On \Gamma_1,\\    
    \v_1 &=  0 \On \partial Y_1\backslash\Gamma_1,  \\
    \int_{Y_0}\alpha_1\D x &= 0.
    \end{aligned}
    \right.
\] 
For $i,j\in\{0,1\}$, let us denote by    
$\mathcal{T}^{ij}:\dot H^{\frac{1}{2}}_{00}(\Gamma_j)\to
H^{-\frac{1}{2}}_{00}(\Gamma_i)$ the Dirichlet-to-Neumann (DtN) operator  
\[
    \mathcal{T}^{ij}[\pphi]:=\sigma(\mathcal{E}_j[\pphi],\mathcal{A}_j[\pphi])\e_1\big\vert_{\Gamma_i}.
\] 
    
The operators $\mathcal{E}_0$ and $\mathcal{E}_1$ can be used to construct Stokes
solutions in $Y_0\cup Y_1$. In order to do this, we need an additional operator     
$\delta \mathcal{Q}$ which computes the pressure difference between two cells.    
\begin{definition}
    We denote by $\delta \mathcal{Q}$ the operator  
    \begin{equation}
    \label{eqn:rfy49}
        \delta \mathcal{Q}[\w_0,\w_2]:=\int_{Y_1}q\D x-\int_{Y_0}q\D x
    \end{equation}
    where $(\w,q)$ is a solution to        
    \begin{equation}
    \label{eqn:c3i38}
    \left\{\begin{aligned}
            -\Delta \w+\nabla q &= 0 \In Y_0\cup Y_1,\\ 
            \div(\w) &= 0 \In Y_0\cup\Y_1,\\  
            \w &= 0 \On \partial(Y_0\cup \partial Y_1)\backslash
            (\Gamma_0\cup\Gamma_1\cup\Gamma_2),\\   
            \w& = \w_0 \On\Gamma_0,\\   
            \w& = \w_2\On\Gamma_2.
    \end{aligned}
    \right.
    \end{equation}
    The operator $\mathcal{\delta Q}$ does not depend on the additive constant
    determining
    the pressure $q$. There exists then a constant $c\in\R$ such that
    \[
    \w= \left\{\begin{aligned}
            \mathcal{E}_0[\w_0]+\mathcal{E}_1[\w|_{\Gamma_1}] \In Y_0,\\
            \tau\mathcal{E}_0\tau^{-1}[\w|_{\Gamma_1}]+\tau\mathcal{E}_1\tau^{-1}[\w_2] \In Y_1,
    \end{aligned}
    \right.\qquad   
    q = c+\left\{\begin{aligned}
            \mathcal{A}_0[\w_0]+\mathcal{A}_1[\w|_{\Gamma_1}] \In Y_0,\\
            \tau\mathcal{A}_0\tau^{-1}[\w|_{\Gamma_1}]+\tau\mathcal{A}_1\tau^{-1}[\w_2] + \delta
            Q[\w_0,\w_2] \In Y_1,
    \end{aligned}
    \right.
    \] 
    Moreover, we have the identity  
    \begin{equation}
    \label{eqn:xvsp4}
    \delta Q[\w_0,\w_2]\e_1 = \tau\mathcal{T}^{00}\tau^{-1}[\w_{|\Gamma_1}] +\tau
    \mathcal{T}^{01}\tau^{-1}[\w_2] -
    \mathcal{T}^{10}[\w_0]-\mathcal{T}^{11}[\w_{|\Gamma_1}] \On \Gamma_1.
    \end{equation}
    Finally, we denote by $\delta Q_0$ and $\delta Q_2$ the operators   
    \begin{equation}
    \label{eqn:6zplh}
        \delta Q_0[\w_0]:=\delta Q[\w_0,0], \qquad \delta Q_2[\w_2]:=\delta
        Q[0,\w_2],
    \end{equation}
    such that by construction, $\delta Q[\w_0,\w_2]=\delta Q_0[\w_0]+\delta
    Q_2[\w_2]$.
\end{definition}
\begin{proof}
The last identity is just the the continuity of the normal flux:
\[
    \begin{aligned}
        \sigma(\mathcal{E}_0[\w_0],\mathcal{A}_0[\w_0])\e_1& +\sigma(\mathcal{E}_1[\w|_{\Gamma_1}],\mathcal{A}_1[\w|_{\Gamma_1}])\e_1   
     \\ 
     &=
    \tau\sigma(\mathcal{E}_0[\tau^{-1}[\w|_{\Gamma_1}]],\mathcal{A}_0[\tau^{-1}\w|_{\Gamma_1}])\e_1+\tau\sigma(\mathcal{E}_1[\tau^{-1}  
    \w_{2}],\mathcal{A}_1[\tau^{-1}\w_{2}])\e_1 \\    
    & \quad-\delta Q[\w_0,\w_2] \e_1 \On \Gamma_1.
    \end{aligned}
\] 
\end{proof}
\begin{remark}
The derivation of the Ricatti equation for the operator $\mathcal{P}$ with the
standard method of \cite{fliss_dirichlet--neumann_2013} does not work
straightforwardly because it creates a term depending on $\delta Q[\w_0,\w_2]$. 
    During our discussion at ENSTA, we considered 
$\z\in H^{1}(Y_1\cup Y_2)$ be any function satisfying   
\[
    \div(\z) &= \left\{\begin{aligned}
            -1 & \In Y_0,\\ 
            1 & \In Y_1,
    \end{aligned}
\right.\qquad \z =0\On \partial (Y_0\cup Y_1).
\]
Due to the condition $\int_{Y_0}\mathcal{A}_0[\w]\D x =0$ for any $\w\in\dotH$, it is
readily seen that   
\[
    \int_{Y_0}\nabla \mathcal{E}_i[\pphi]\cdot \nabla\z\D x =
    \int_{\Gamma_1}\sigma(\mathcal{E}_i[\pphi],\mathcal{A}_i[\pphi])\e_1\cdot\z\D
    \sigma=\int_{\Gamma_1} \mathcal{T}^{1i}[\pphi]\cdot\z\D \sigma,
\] 
\[
    \int_{Y_1}\nabla (\tau\mathcal{E}_i[\pphi])\cdot \nabla\z\D x =
    -\int_{\Gamma_1}\tau\sigma(\mathcal{E}_i[\pphi],\mathcal{A}_i[\pphi])\e_1\cdot\z\D
    \sigma=-\int_{\Gamma_1} \tau\mathcal{T}^{0i}[\pphi]\cdot\z\D \sigma.
\] 
Multiplying the first equation of \cref{eqn:c3i38} by $\z$ and integrating by parts,
we then found    that the pressure difference is given by 
\[
    \begin{aligned}
    \delta \mathcal{Q}[\w_0,\w_2]  & = \int_{Y_0\cup Y_1} q\div(\z)\D x 
      =  
    \int_{Y_0\cup Y_1}\nabla \w\cdot\nabla \z\D x    
    \\
     & =\int_{Y_0} (\nabla \mathcal{E}_0[\w_0]+ \nabla\mathcal{E}_1[\w|_{\Gamma_1}])\cdot \nabla \z   
     +\int_{Y_1}\nabla(\tau\mathcal{E}_0\tau^{-1}[\w|_{\Gamma_1}]+\tau
     \mathcal{E}_1\tau^{-1}[\w_2])\cdot\nabla\z\D x\\    
     &= \int_{\Gamma_1} (\mathcal{T}^{10}[\w_0]+\mathcal{T}^{11}[\w|_{\Gamma_1}])\cdot\z\D
     \sigma -\int_{\Gamma_1}(\tau\mathcal{T}^{00}\tau^{-1}[\w|_{\Gamma_1}] +
     \tau\mathcal{T}^{01}\tau^{-1}[\w_2])\cdot\z\D x.   
    \end{aligned}
\] 
Inserting these expression, we could thus replace $\delta Q[\w_0,\w_2]$ by a linear
functional in $\w_0$, $\w_{|\Gamma_1}$ and $\w_2$ in \cref{eqn:xvsp4} to read a
Ricatti equation for $\mathcal{P}$. In fact, this computation is just a consequence
of  
\[
\int_{\Gamma_1}\e_1 \cdot\z \D \sigma = \int_{\partial Y_0}\n\cdot\z\D \sigma =
\int_{Y_0}\div(\z)\D x= -1.
\] 
Therefore, a more straightfoward manner to rewrite $\delta Q[\w_0,\w_2]$ is to
integrate \cref{eqn:xvsp4} against $\e_1$, which yields the next lemma.
\end{remark}
\begin{lemma}
    The traces $\w_0$, $\w_1$ and $\w_2$ are related by the formula     
    \begin{equation}
    \label{eqn:ible6}
        -\Pi\mathcal{T}^{10}[\w_0]+(\Pi\tau\mathcal{T}^{00}\tau^{-1}-\Pi\mathcal{T}^{11})[\w|_{\Gamma_1}]+\Pi\tau\mathcal{T}^{01}\tau^{-1}[\w_2]=0,
    \end{equation}
    where   $ \Pi$ is the projection operator 
    \[
        \Pi[\pphi]:= \pphi-\left(\int_{\Gamma_1}\pphi\cdot\e_1\D \sigma\right)\e_1.
    \] 
\end{lemma}
\begin{proof}
    Multiplying \cref{eqn:xvsp4} by $\e_1$ and integrating on $\Gamma_1$, we read   
    \[
        \delta Q[\w_0,\w_2] = \int_{\Gamma_1}\left( \tau
        \mathcal{T}^{00}\tau^{-1}[\w|_{\Gamma_1}]-\mathcal{T}^{11}[\w|_{\Gamma_1}]+\tau
    \mathcal{T}^{01}\tau^{-1}[\w_2]-\mathcal{T}^{10}[\w_0] \right)\cdot\e_1\D \sigma
    \] 
\end{proof}
   \begin{remark}
    I think we should be able to prove that $\Pi\tau
    \mathcal{T}^{00}\tau^{-1}-\Pi\mathcal{T}^{11}$ is an invertible operator as the
    sum of two symmetric positive definite operators. 
   \end{remark} 
   \begin{corollary}
       The propagator operator $\mathcal{P}$ solves the following Ricatti equation.     
       \begin{equation}
       \label{eqn:1p64i}
        -\Pi\tau^{-1}\mathcal{T}^{10}+(\Pi\mathcal{T}^{00}-\Pi\tau^{-1}\mathcal{T}^{11}\tau)\mathcal{P}+\Pi\mathcal{T}^{01}\tau\mathcal{P}^{2}=0,
       \end{equation}
   $\mathcal{P}$ is the unique operator with
   spectral radius $\sigma<1$ solving the Ricatti equation \cref{eqn:1p64i}.
   \end{corollary}
   \begin{proof}
       If $\mathcal{P}$ is given, taking $\w_2:=\tau^{2}\mathcal{P}^{2}$ implies
       $\w|_{\Gamma_1}=\mathcal{P}[\w_0]$. Then, \cref{eqn:ible6} with arbitrary
       $\w_0\in\dotH$ implies
       \cref{eqn:1p64i}, noting that $\Pi$ et $\tau$ commute. Reciprocally, if
       $\mathcal{P}$ is such an operator, we define a solution $(\v,q)\in
       H^{1}(\mathcal{B}_+,\R^{d})\times L^{2}(\mathcal{B}_+)/\R$ by setting    
       \begin{equation}
       \label{eqn:pr9ux}
           \left\{\begin{aligned}
           \v & =
           \mathcal{E}_0[\mathcal{P}^{k}[\w_0]]+\mathcal{E}_1[\tau\mathcal{P}^{k+1}[\w_0]],\\
q     & =\mathcal{A}_0[\mathcal{P}^{k}[\w_0]]+\mathcal{A}_1[\tau\mathcal{P}^{k+1}[\w_0]]+\sum_{i=0}^{k-1}
           \delta Q[\mathcal{P}^{i}[\w_0],\tau^{2}\mathcal{P}^{i+2}[\w_0]]+c,
           \end{aligned}
           \right.\qquad  \In Y_k.
       \end{equation}
       for some constant $c\in\R$. 
       By construction of $\mathcal{P}$, $(\w,q)\in H^{1}(\mathcal{B}_+,\R^{d})\times
       L^{2}(\mathcal{B}_+)/\R$ is the solution to the semi-strip
       problem with Dirichlet datum $\w_0$. Since $\v|_{\Gamma_1}=\mathcal{P}[\w_0]$,
       it follows that $\mathcal{P}$ is the propagation operator.
   \end{proof}
   We conclude this section by noting that the representation \cref{eqn:pr9ux} gives
   a mean to compute boundary layer tails for the pressure.     
   \begin{proposition}
       The constant $c$ in the representation \cref{eqn:pr9ux} so that 
   \[
   q(x_1,x')\rightarrow 0 \text{ as }x_1\rightarrow +\infty. 
   \] 
   is given explicitly by 
   \[
   c=-\big(\delta Q_0 (I-\mathcal{P})^{-1} +\delta Q_2\tau^{2}(I-\mathcal{P})^{-1}
      \mathcal{P}^{2}\big)[\w_0]
   \] 
   where we recall the definition \cref{eqn:6zplh} of the operators $\delta Q_0$ et
   $\delta Q_2$.
   \end{proposition}    
   \begin{proof}
  Since $\mathcal{P}$ has spectral radius $\sigma<1$, the terms
  $\mathcal{A}_0[\mathcal{P}^{k}[\w_0]]$ and $\mathcal{A}_1[\tau
  \mathcal{P}^{k+1}[\w_0]]$ decay geometrically. We have therefore  
  \[
      \begin{aligned}
      c & =-\sum_{i=0}^{+\infty}\delta
      Q[\mathcal{P}^{i}[\w_0],\tau^{2}\mathcal{P}^{i+2}[\w_0]]  
      = -\sum_{i=0}^{+\infty} (\delta Q_0 \mathcal{P}^{i} +\delta Q_2
      \tau^{2}\mathcal{P}^{i+2})[\w_0]\\    
      &=-\big(\delta Q_0 (I-\mathcal{P})^{-1} +\delta Q_2\tau^{2}(I-\mathcal{P})^{-1}
      \mathcal{P}^{2}\big)[\w_0].
      \end{aligned}
  \] 
   \end{proof}    
\subsection{Alternative derivation of a Ricatti equation}
Inspired by the methodology of the previous section, we obtain the intuition that
a simpler scheme might exist to derive a Ricatti equation characterizing the
propagation operator $\mathcal{P}$. Let us consider the operator
$\mathcal{V}[\w_0,\w_2]$ defined for $\w_0\in \dotH$ and $\w_2\in \dot
H^{\frac{1}{2}}(\Gamma_2,\R^{d})$ by    
\begin{equation}
\label{eqn:88eqm}   
\mathcal{V}[\w_0,\w_2]=\w,
\end{equation}  
where $(\w,q)\in H^{1}(Y_0\cup Y_1,\R^{d})\times L^{2}(Y_0\cup Y_1)/\R$ is the unique solution to \cref{eqn:c3i38}. 
By using the linearity, we may also define $\mathcal{V}_0$ and $\mathcal{V}_2$ the operators such that  
\[
    \mathcal{V}[\w_0,\w_2]=\mathcal{V}_0[\w_0]+\mathcal{V}_2[\w_2],
\]
namely  
\[
    \mathcal{V}_0[\w_0]:=\mathcal{V}[\w_0,0],\qquad
    \mathcal{V}_2[\w_2]:=\mathcal{V}[0,\w_2].
\] 
Then, we must obviously have    
\[
    \mathcal{V}[\w_0,\tau^{2}\mathcal{P}^{2}[\w_0]]=\tau\mathcal{P} [\w_0] \On \Gamma_1,
\] 
namely  
\begin{equation}
\label{eqn:0hn7g}
\tau^{-1}\gamma\mathcal{V}_0-\mathcal{P}+\tau^{-1}\gamma\mathcal{V}_2\tau^{2} \mathcal{P}^{2}=0.
\end{equation}
where $\gamma$ is the trace operator on $\Gamma_1$.
\section{Numerical results} 
\subsection{Laplace equation}   
Let us first consider the computation of solutions for the Laplace equation in
perforated semi-strip problems, namely  given $u_0\in
H^{\frac{1}{2}}(\Gamma_0)$,    
find $u\in H^{1}(\mathcal{B}_+)$ such that
\begin{equation}
\label{eqn:q7aj0}   
\left\{\begin{aligned}
        -\Delta u &= 0 \In \mathcal{B}_+,\\ 
        u &= u_0 \On \Gamma_0.
\end{aligned}
\right.
\end{equation}
For an illustrative purpose, we consider either Neumann or Dirichlet boundary
conditions on the top and bottom boundaries:    
\begin{equation}
\label{eqn:4cxg3}
        u = 0 \On \partial \mathcal{B}_+\backslash \Gamma_0 \text{ (Dirichlet)},
\end{equation}
\begin{equation}
\label{eqn:4e0hf}
    \partialn{u} = 0 \On \partial \mathcal{B}_+\backslash \Gamma_0 \text{ (Neumann)}.
\end{equation}
When considering Dirichlet boundary conditions, $u_0$ should be in
$H^{\frac{1}{2}}_{00}(\Gamma_0)$. We shall verify the slower exponential decays in case of
Neumann boundary conditions on the top and bottom wall. 
    
\medskip    

We consider the following fixed point scheme to compute the operator $\mathcal{P}$
based on \cref{eqn:0hn7g}:  
\begin{equation}
\label{eqn:08maf}
\mathcal{P}_{k+1}=\tau^{-1}\gamma \mathcal{V}_0+\tau^{-1}\gamma
\mathcal{V}_2\tau^{2}\mathcal{P}_k^{2}, \qquad k=0,1,2,\dots.
\end{equation}
{\color{blue} This needs a proof,} but we may expect the mapping
$f(\mathcal{P}):=\tau^{-1}\gamma \mathcal{V}_0+\tau^{-1}\gamma
\mathcal{V}_2\tau^{2}\mathcal{P}^{2}$ to be a contraction in a neighborhood of zero.
Indeed,     very heuristically and assuming $\mathcal{P}_1$ and $\mathcal{P}_2$
commute, 
\[
f(\mathcal{P}_1)-f(\mathcal{P}_2)=\tau^{-1}\gamma
\mathcal{V}_2\tau^{2}(\mathcal{P}_1^{2}-\mathcal{P}_2^{2}) = \tau^{-1}\gamma
\mathcal{V}_2\tau^{2}(\mathcal{P}_1+\mathcal{P}_2)(\mathcal{P}_1-\mathcal{P}_2).
\] 
Therefore, the Lipschitz constant of $f$ is bounded by the norms of the operator
$\tau^{-1}\gamma\mathcal{V}_2$ multiplied by a small value for $\mathcal{P}_1+\mathcal{P}_2$. Due to
the decay of the solution in the double cell, it can be expected that $|||\tau^{-1}\gamma
\mathcal{V}_2|||<1$.
    
\medskip    

In practice, the scheme \cref{eqn:08maf} seems to work well, returning the unique
solution to \cref{eqn:0hn7g} of spectral radius smaller than one. The examples tested   
reuqired only 4 to 12 
iterations to reach convergence on the example tested. 
    
\bigskip    

In the next \cref{fig:disk_Dirichlet} to \cref{fig:disks_Neumann}, we plot the Floquet modes
associated to the boundary layers for the Laplace semi-strip problems
\cref{eqn:q7aj0} with Dirichlet and Neumann boundary conditions
\cref{eqn:4cxg3,eqn:4e0hf} on the top and bottom walls. We consider different types
of geometries: a disk, a triangle, a group of disks.  For each case, we do the
following operations:   
\begin{enumerate}
    \item we compute the operator $\mathcal{V}_0$ and $\mathcal{V}_2$ by solving
        Laplace problems in the double cell with multiple right-hand sides;     
    \item we compute the operator $\mathcal{P}$ by iterating \cref{eqn:08maf};  
    \item we diagonalize the operator $\mathcal{P}$ (a dense matrix) using the NumPy
        function \texttt{eigs}. We order the eigenvalues $(\mu_k)_{k\in\N}$ 
        according to their real parts. {\color{blue} Strikingly, we observe that the
        eigenvalues of $\mathcal{P}$ are always real (although $\mathcal{P}$ is a
        priori not
    a symmetric operator)!} We denote by $(\phi_k)_{k\in\N}$ the corresponding
    eigenvectors.
    \item We define the Floquet multipliers   $(\lambda_k)_{k\in\N}$  and Floquet
        modes $(u_k)_{k\in\N}$ by
        \[
        \lambda_k := \log(\mu_k); \qquad    
        u_k(x_1,x'):=
        (\mathcal{V}_0[\phi_k](x_1,x')+\mathcal{V}_2[\mathcal{P}^{2}\phi_k](x_1,x'))e^{-\lambda_k
        x_1}.
        \] 
        With these definitions, we expect asymptotic expansions (assuming
        $\mathcal{P}$ has only eigenvalues of multiplicity one)         
        \[
        u(x)=\sum_{k=0}^{+\infty} c_k e^{\lambda_k x_1}u_k(x_1,x'), \qquad \text{ if
        }u_0(x)=\sum_{k=0}^{+\infty}c_k \phi_k.
        \]  
        Since by construction $e^{\lambda_k}=\mu_k$ are the eigenvalues of
        $\mathcal{P}$, the modes $u_k(x_1,x')$ are periodic, which is verified on
        \cref{fig:disk_Dirichlet} to \cref{fig:disks_Neumann}. 
\end{enumerate}
    
    
    
        \begin{figure}  
        \centering  
           
            \begin{subfigure}{0.3\linewidth}    
            \includegraphics[width=\linewidth]{FIGS/eigenmodes_disk_Dirichlet_0.png}   
            \caption{$\lambda_{0}=-5.31}
\end{subfigure}\quad             
            \begin{subfigure}{0.3\linewidth}    
            \includegraphics[width=\linewidth]{FIGS/eigenmodes_disk_Dirichlet_1.png}   
            \caption{$\lambda_{1}=-6.69}
\end{subfigure}\quad             
            \begin{subfigure}{0.3\linewidth}    
            \includegraphics[width=\linewidth]{FIGS/eigenmodes_disk_Dirichlet_2.png}   
            \caption{$\lambda_{2}=-11.6}
\end{subfigure}\quad             
            \begin{subfigure}{0.3\linewidth}    
            \includegraphics[width=\linewidth]{FIGS/eigenmodes_disk_Dirichlet_3.png}   
            \caption{$\lambda_{3}=-13.4}
\end{subfigure}\quad             
            \begin{subfigure}{0.3\linewidth}    
            \includegraphics[width=\linewidth]{FIGS/eigenmodes_disk_Dirichlet_4.png}   
            \caption{$\lambda_{4}=-18.2}
\end{subfigure}\quad             
        \caption{Modes for a disk with Dirichlet boundary conditions on   
        the top and bottom wall.}
        \label{fig:disk_Dirichlet}
        \end{figure}    
        \begin{figure}  
        \centering  
           
            \begin{subfigure}{0.3\linewidth}    
            \includegraphics[width=\linewidth]{FIGS/eigenmodes_disk_Neumann_0.png}   
            \caption{$\lambda_{0}=-2.50}
\end{subfigure}\quad             
            \begin{subfigure}{0.3\linewidth}    
            \includegraphics[width=\linewidth]{FIGS/eigenmodes_disk_Neumann_1.png}   
            \caption{$\lambda_{1}=-3.34}
\end{subfigure}\quad             
            \begin{subfigure}{0.3\linewidth}    
            \includegraphics[width=\linewidth]{FIGS/eigenmodes_disk_Neumann_2.png}   
            \caption{$\lambda_{2}=-8.40}
\end{subfigure}\quad             
            \begin{subfigure}{0.3\linewidth}    
            \includegraphics[width=\linewidth]{FIGS/eigenmodes_disk_Neumann_3.png}   
            \caption{$\lambda_{3}=-10.0}
\end{subfigure}\quad             
            \begin{subfigure}{0.3\linewidth}    
            \includegraphics[width=\linewidth]{FIGS/eigenmodes_disk_Neumann_4.png}   
            \caption{$\lambda_{4}=-14.9}
\end{subfigure}\quad             
        \caption{Modes for a disk with Neumann boundary conditions on   
        the top and bottom wall.}
        \label{fig:disk_Neumann}
        \end{figure}    
        \begin{figure}  
        \centering  
           
            \begin{subfigure}{0.3\linewidth}    
            \includegraphics[width=\linewidth]{FIGS/eigenmodes_triangle_Dirichlet_0.png}   
            \caption{$\lambda_{0}=-6.77}
\end{subfigure}\quad             
            \begin{subfigure}{0.3\linewidth}    
            \includegraphics[width=\linewidth]{FIGS/eigenmodes_triangle_Dirichlet_1.png}   
            \caption{$\lambda_{1}=-9.28}
\end{subfigure}\quad             
            \begin{subfigure}{0.3\linewidth}    
            \includegraphics[width=\linewidth]{FIGS/eigenmodes_triangle_Dirichlet_2.png}   
            \caption{$\lambda_{2}=-13.7}
\end{subfigure}\quad             
            \begin{subfigure}{0.3\linewidth}    
            \includegraphics[width=\linewidth]{FIGS/eigenmodes_triangle_Dirichlet_3.png}   
            \caption{$\lambda_{3}=-18.7}
\end{subfigure}\quad             
            \begin{subfigure}{0.3\linewidth}    
            \includegraphics[width=\linewidth]{FIGS/eigenmodes_triangle_Dirichlet_4.png}   
            \caption{$\lambda_{4}=-20.9}
\end{subfigure}\quad             
        \caption{Modes for a triangle with Dirichlet boundary conditions on   
        the top and bottom wall.}
        \label{fig:triangle_Dirichlet}
        \end{figure}    
        \begin{figure}  
        \centering  
           
            \begin{subfigure}{0.3\linewidth}    
            \includegraphics[width=\linewidth]{FIGS/eigenmodes_triangle_Neumann_0.png}   
            \caption{$\lambda_{0}=-3.36}
\end{subfigure}\quad             
            \begin{subfigure}{0.3\linewidth}    
            \includegraphics[width=\linewidth]{FIGS/eigenmodes_triangle_Neumann_1.png}   
            \caption{$\lambda_{1}=-4.62}
\end{subfigure}\quad             
            \begin{subfigure}{0.3\linewidth}    
            \includegraphics[width=\linewidth]{FIGS/eigenmodes_triangle_Neumann_2.png}   
            \caption{$\lambda_{2}=-10.2}
\end{subfigure}\quad             
            \begin{subfigure}{0.3\linewidth}    
            \includegraphics[width=\linewidth]{FIGS/eigenmodes_triangle_Neumann_3.png}   
            \caption{$\lambda_{3}=-14.0}
\end{subfigure}\quad             
            \begin{subfigure}{0.3\linewidth}    
            \includegraphics[width=\linewidth]{FIGS/eigenmodes_triangle_Neumann_4.png}   
            \caption{$\lambda_{4}=-17.3}
\end{subfigure}\quad             
        \caption{Modes for a triangle with Neumann boundary conditions on   
        the top and bottom wall.}
        \label{fig:triangle_Neumann}
        \end{figure}    
        \begin{figure}  
        \centering  
           
            \begin{subfigure}{0.3\linewidth}    
            \includegraphics[width=\linewidth]{FIGS/eigenmodes_disks_Dirichlet_0.png}   
            \caption{$\lambda_{0}=-9.21}
\end{subfigure}\quad             
            \begin{subfigure}{0.3\linewidth}    
            \includegraphics[width=\linewidth]{FIGS/eigenmodes_disks_Dirichlet_1.png}   
            \caption{$\lambda_{1}=-9.98}
\end{subfigure}\quad             
            \begin{subfigure}{0.3\linewidth}    
            \includegraphics[width=\linewidth]{FIGS/eigenmodes_disks_Dirichlet_2.png}   
            \caption{$\lambda_{2}=-13.0}
\end{subfigure}\quad             
            \begin{subfigure}{0.3\linewidth}    
            \includegraphics[width=\linewidth]{FIGS/eigenmodes_disks_Dirichlet_3.png}   
            \caption{$\lambda_{3}=-17.6}
\end{subfigure}\quad             
            \begin{subfigure}{0.3\linewidth}    
            \includegraphics[width=\linewidth]{FIGS/eigenmodes_disks_Dirichlet_4.png}   
            \caption{$\lambda_{4}=-22.5}
\end{subfigure}\quad             
        \caption{Modes for a disks with Dirichlet boundary conditions on   
        the top and bottom wall.}
        \label{fig:disks_Dirichlet}
        \end{figure}    
        \begin{figure}  
        \centering  
           
            \begin{subfigure}{0.3\linewidth}    
            \includegraphics[width=\linewidth]{FIGS/eigenmodes_disks_Neumann_0.png}   
            \caption{$\lambda_{0}=-4.89}
\end{subfigure}\quad             
            \begin{subfigure}{0.3\linewidth}    
            \includegraphics[width=\linewidth]{FIGS/eigenmodes_disks_Neumann_1.png}   
            \caption{$\lambda_{1}=-4.99}
\end{subfigure}\quad             
            \begin{subfigure}{0.3\linewidth}    
            \includegraphics[width=\linewidth]{FIGS/eigenmodes_disks_Neumann_2.png}   
            \caption{$\lambda_{2}=-11.1}
\end{subfigure}\quad             
            \begin{subfigure}{0.3\linewidth}    
            \includegraphics[width=\linewidth]{FIGS/eigenmodes_disks_Neumann_3.png}   
            \caption{$\lambda_{3}=-14.7}
\end{subfigure}\quad             
            \begin{subfigure}{0.3\linewidth}    
            \includegraphics[width=\linewidth]{FIGS/eigenmodes_disks_Neumann_4.png}   
            \caption{$\lambda_{4}=-17.4}
\end{subfigure}\quad             
        \caption{Modes for a disks with Neumann boundary conditions on   
        the top and bottom wall.}
        \label{fig:disks_Neumann}
        \end{figure}
\bibliography{references,fliss_feppon}
\bibliographystyle{acm}
\end{document}
